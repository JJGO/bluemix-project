\documentclass[a4paper]{article}
  \input{_core.tex}
  \usepackage{parskip}
  % \addbibresource{mybib.bib}
    %\usepackage{nameref}
    %\usepackage{tabularx}
  % \usepackage{array}
  \usepackage{fancyhdr}
  \lhead{Arquitectura de Sistemas en Red}
  \rhead{Desarrollo de una aplicación con Bluemix}

% Global variables
  \newcommand{\HWTitle}{Desarrollo de una aplicación con Bluemix}
  \newcommand{\HWDueDate}{\today}
  \newcommand{\HWAuthorName}{José Javier González Ortiz \and Lucía Montero Sanchis}
  \title{\HWTitle \\ \vspace{.25cm}}
  \date{\HWDueDate}
  \author{\HWAuthorName}

\usepackage{graphicx}
\usepackage{wrapfig}
\usepackage{lscape}
\usepackage{rotating}
\usepackage{epstopdf}

\begin{document}
\maketitle
\pagestyle{fancy}
\vfill
%\listoffigures
\vfill\vfill
\newpage
\section{Introducción} % (fold)
\label{sec:introducción}
  Se ha desarrollado una aplicación Flask basada en Python, con los siguientes cinco servicios:
  \begin{itemize}
    \item Language Translator
    \item Visual Recognition
    \item Text to Speech
    \item Natural Language Understanding
    \item Cloudant NoSQL DB
  \end{itemize}
  El objetivo de la aplicación es ofrecer un servicio de análisis de imágenes y de frases en diferentes idiomas.

  La aplicación usa cookies para mantener historial de usuarios de forma independiente. Se soporta también url estáticas para búsquedas ya realizadas. Esto permite poder compartir o salvar búsquedas previas.

  Se ha empleado AJAX para las pantallas de carga y para obtener el historial de los usuarios. La aplicación soporta cacheado de resultados de cara a evitar repetir peticiones en un espacio de tiempo corto debido al alto numero de comunicaciones de microservicios presentes. Esto agiliza mucho consultas del historial.

\begin{figure}[htp!]
    \centering
    \caption{La aplicación está disponible tanto en Bluemix como en una página web personal. El código está compartido en GitHub, y se ha utilizado TravisCI para la Integración Continua.}
    \label{fig:webs}
    \includegraphics[width=\textwidth]{webs1}
\end{figure}
\section{Enlaces URL} % (fold)
\label{sec:enlaces_url}

% section enlaces_url (end)

\begin{figure}[htp!]
    \centering
    \caption{Bluemix: \url{http://erittely.josejg.com} \newline{} \url{http://erittely.eu-gb.mybluemix.net/} }
    \label{fig:blue}
    \includegraphics[width=\textwidth]{blue}
\end{figure}


\begin{figure}[htp!]
    \centering
    \caption{GitHub: \url{https://github.com/JJGO/bluemix-project}}
    \label{fig:git}
    \includegraphics[width=\textwidth]{git}
\end{figure}

\begin{figure}[htp!]
    \centering
    \caption{Travis: \url{https://travis-ci.org/JJGO/bluemix-project}}
    \label{fig:travis}
    \includegraphics[width=\textwidth]{travis}
\end{figure}

\begin{figure}[htp!]
    \centering
    \caption{El uso de Integración Continua queda reflejado en GitHub, donde el código aparece como 'build:passing'}
    \label{fig:git2}
    \includegraphics[width=\textwidth]{git2}
\end{figure}


% section introducción (end)
\newpage
\clearpage
\section{Capturas de pantalla de la aplicación} % (fold)
\label{sec:capturas_de_pantalla_de_la_aplicación}
\subsection{Responsividad} % (fold)
\label{sub:responsividad}

% subsection responsividad (end)
\begin{figure}[htp!]
    \centering
    \caption{Pantalla de inicio (responsividad I)}
    \label{fig:1}
    \includegraphics[width=0.7\textwidth]{1}
\end{figure}
\begin{figure}[htp!]
    \centering
    \caption{Pantalla de inicio (responsividad II)}
    \label{fig:2}
    \includegraphics[width=0.5\textwidth]{2}
\end{figure}
\begin{figure}[htp!]
    \centering
    \caption{Pantalla de inicio (responsividad III)}
    \label{fig:3}
    \includegraphics[width=0.45\textwidth]{3}
\end{figure}
\newpage
\clearpage
\subsection{Funcionamiento de la aplicación} % (fold)
\label{sub:funcionamiento_de_la_aplicación}

% subsection funcionamiento_de_la_aplicación (end)
\begin{figure}[htp!]
    \centering
    \caption{(1) Introducir frase}
    \label{fig:5}
    \includegraphics[width=\textwidth]{a}
\end{figure}
\begin{figure}[htp!]
    \centering
    \caption{(2) Procesamiento de la frase}
    \label{fig:6}
    \includegraphics[width=\textwidth]{b}
\end{figure}
\begin{figure}[htp!]
    \centering
    \caption{(3) Resultado. A la derecha aparece el historial, en el que ahora se incluye esta frase.}
    \label{fig:7}
    \includegraphics[width=\textwidth]{c}
\end{figure}
\begin{figure}[htp!]
    \centering
    \caption{(4) Ahora analizamos una imagen}
    \label{fig:9}
    \includegraphics[width=\textwidth]{d2}
\end{figure}
\begin{figure}[htp!]
    \centering
    \caption{(5) Resultado obtenido. Se ha añadido este elemento en el historial.}
    \label{fig:10}
    \includegraphics[width=\textwidth]{e2}
\end{figure}
\begin{figure}[htp!]
    \centering
    \caption{(6) Se puede hacer click en cualquier elemento del historial para repetir consultas anteriores}
    \label{fig:11}
    \includegraphics[width=\textwidth]{f2}
\end{figure}

\begin{figure}[htp!]
    \centering
    \caption{(7) Se repite la consulta anterior, y el elemento correspondiente del historial se coloca como el más reciente}
    \label{fig:11}
    \includegraphics[width=\textwidth]{g}
\end{figure}
% section capturas_de_pantalla_de_la_aplicación (end)
\end{document}
